\documentclass[a4paper, 10pt]{article}
\usepackage[UTF8]{ctex}
\usepackage{geometry}
\usepackage{graphicx}
\usepackage{setspace}
\usepackage{float}
\usepackage{listings}
\usepackage{xcolor}
\usepackage{multirow}
\lstset{
    numbers=left, 
    numberstyle= \tiny, 
    keywordstyle= \color{ blue!70},
    commentstyle= \color{red!50!green!50!blue!50}, 
    frame=shadowbox, % 阴影效果
    rulesepcolor= \color{ red!20!green!20!blue!20} ,
    escapeinside=``, % 英文分号中可写入中文
    xleftmargin=2em,xrightmargin=2em, aboveskip=1em,
    framexleftmargin=2em
} 
\geometry{left = 1.0 cm, right = 1.0cm, top = 2.0cm, bottom = 2.0cm	}
\title{编译原理第五章(二)}
\author{李鹏辉}

\begin{document}
\maketitle 

1.(5.3.1):下面是设计运算符+和整数或浮点运算分量的表达式的文法。区分浮点数的方法是看它有无小数点

$E \rightarrow E+T | T$

$T \rightarrow num.num | num$

1)给出一个SDD来确定每个项T和表达式E的类型

\begin{table}[H]
\centering
\begin{tabular}{c|c}
\hline
\hline
产生式 & 规则\\
\hline
$E \rightarrow E_1 + T$ &$E.t = (E_1.t == Float || T.t == Float)? Float : Integer;$  \\
$E\rightarrow T$ & $E.t = T.t$\\
$T \rightarrow num.num$ &$T.t = Float$ \\
$T\rightarrow num$ & $T.t = Integer$\\
\hline
\end{tabular}
\end{table}

2)扩展1)中得到的SDD,使它可以把表达式转换为后缀表达式。使用一个单目运算符intToFloat把一个整数转换为相等的浮点数。
\begin{table}[H]
\centering
\begin{tabular}{|c|c|}
\hline
\hline
产生式 & 规则\\
\hline
\multirow{5}{*}{$E \rightarrow E_1 + T$} &$E.t = (E_1.t == Float || T.t == Float)? Float : Integer;$  \\
 	& $  E = (E_1.t  == Integer \&\& T.t == Float)? intToFloat(E_1.n)||T.n||'+' : $ \\
& $(E_1.t == Float \&\&T.t == Integer)? E_1.n || intToFloat(T.n) ||'+' : $\\
& $E_1.n || T.n || '+'; $\\
\hline
$E\rightarrow T$ & $E.t = T.t, E.n = T.n$\\
\hline
$T \rightarrow num.num$ &$T.t = Float, T.n = num.num$ \\
\hline
$T\rightarrow num$ & $T.t = Integer, T.n = num$\\
\hline
\end{tabular}
\end{table}

2.(5.4.2)改写下面SDT

$A \rightarrow A\{a\}B | AB\{b\} | 0$

$B \rightarrow  B\{c\}A | BA\{d\} | 1$

使得基础文法变成非左递归的。其中a,b,c,d是语义动作,0和1是终结符号。

$A\rightarrow 0A'$

$A' \rightarrow \{a\}BA' | B\{b\}A' | \varepsilon$

$B\rightarrow 1B'$

$B' \rightarrow  \{c\}AB' | A\{d\}B' | \varepsilon$


3.(5.4.6)修改图5-25中SDD,使它包含一个综合属性B.le,即一个Box的长度。两个Box并列后得到的Box长度是这两个Box长度和,然后将你的新规则加入到图5-26中合适的位置上。
\begin{table}[H]
\centering

\begin{tabular}{|c|c|}
\hline
\hline
PRODUCTION & SEMANTIC RULES\\
\hline
\multirow{1}{*}{$S\rightarrow B $} & $B.ps = 10$ \\
\hline
\multirow{5}{*}{$B\rightarrow B_1B_2 $} & $B_1.ps = B.ps$ \\
& $B_2.ps = B.ps$\\
& $B.ht = max(B_1.ht, B_2.ht)$\\
& $B.dp = max(B_1.dp, B_2.dp)$\\
& $B.le = B_1.le + B_2.le$\\
\hline
\multirow{5}{*}{$B\rightarrow B_1 \;sub\; B_2$} & $B_1.ps = B.ps$\\
& $B_2.ps = 0.7 \times B.ps$\\
& $B.ht = max(B_1.ht, B_2.ht - 0.25\times B.ps)$\\
& $B.dp = max(B_1.dp,B_2.dp + 0.25\times B.ps)$\\
& $B.le = B1.e + 0.7\times B_2.le$\\
\hline
\multirow{4}{*}{$B\rightarrow (B_1)$} & $B_1.ps = B.ps$\\
& $B.ht = B_1.ht$\\
& $B.dp = B_1.dp$\\
& $B.le = B_1.le$\\
\hline
\multirow{3}{*}{$B\rightarrow text$} & $B.ht = getHt(B.ps, text.lexval)$\\
&$B.dp = getDp(B.ps, text.lexval)$\\
& $B.le = getLe(B.ps, textlexval)$\\

\hline
\end{tabular}
\end{table}\end{document}
