\documentclass[a4paper, 16pt]{article}
\usepackage[UTF8]{ctex}
\usepackage{geometry}
\usepackage{graphicx}
\usepackage{setspace}
\usepackage{float}
\usepackage{listings}
\usepackage{xcolor}
\lstset{
    numbers=left, 
    numberstyle= \tiny, 
    keywordstyle= \color{ blue!70},
    commentstyle= \color{red!50!green!50!blue!50}, 
    frame=shadowbox, % 阴影效果
    rulesepcolor= \color{ red!20!green!20!blue!20} ,
    escapeinside=``, % 英文分号中可写入中文
    xleftmargin=2em,xrightmargin=2em, aboveskip=1em,
    framexleftmargin=2em
} 
\geometry{left = 1.0 cm, right = 1.0cm, top = 2.0cm, bottom = 2.0cm	}
\title{编译原理第三章(一)}
\author{李鹏辉}

\begin{document}
\maketitle
1. (3.1.1)将下面C++程序划分为正确的词素序列。哪些词素应该有相关联的词法值? 应该具有什么值?\\

\lstset{language=C}
\begin{lstlisting}
float limiteSquare(x){
	float x;
	/*return x-squared, but never than 100*/
	return (x <= -10.0 || x >= 10.0)? 100: x*x;
}
\end{lstlisting}
\begin{table}[H]
\centering
\caption{Analysis}
\begin{tabular}{c|c|c}
\hline
词素 & 记号 & 属性\\
\hline
$float$ & 关键字 & $float$\\
\hline
$limiteSquare$ &标识符 &指向limitSquare的条目指针\\
\hline
( & 标点符号 & (左括号 \\
\hline
x & 标识符 & 指向x的条目指针\\
\hline
) & 标点符号 & )右括号 \\
\hline
$\{$ & 标点符号 &$\{ $花括号\\
\hline
$float$ & 关键字 & $float$\\
\hline
x & 标识符 & 指向x的条目指针\\
\hline
; & 标点符号 & ;分号\\
\hline
return & 关键字 & return \\
\hline
( & 标点符号 & (左括号 \\
\hline
x & 标识符 & 指向x的条目指针\\
\hline
$>= $& 算符 & $>=$ 大于等于\\
\hline
-10.0 &常数 & -10.0\\
\hline
x & 标识符 & 指向x的条目指针\\
\hline
$<=$ & 算符 & $<=$ 小于等于 \\
\hline
10.0 &常数 & 10.0\\
\hline
) & 标点符号 & )右括号 \\
\hline
? & 标点符号& ?问号 \\
\hline
100 & 常数 & 100\\
\hline
x & 标识符 & 指向x的条目指针\\
\hline
* & 算符 & *乘号\\
\hline
x & 标识符 & 指向x的条目指针\\
\hline
; & 标点符号 & ;分号\\
\hline
$\}$ & 标点符号 &$\}$右花括号\\
\hline
\end{tabular}
\end{table}
\bigskip
2.(3.3.2)[1,2,5]描述下列正则表达式定义的语言。\\
1) $a(a|b)^*a$\\
字母表为$\{a,b\}$,以字母a开头和结尾,中间部分是长度$>=0$任意串。\\
\\
2) $((\varepsilon|a)b^*)^*$\\
字母表为$\{a,b\}$的任意有限长度串。\\
\\
3) $(aa|bb)^*((ab|ba)(aa|bb)^*(ab|ba)(aa|bb)^*)^*$\\
字母表为$\{a,b\}$的偶数个a和b的全体字符串。\\
\bigskip
\\
3.(3.3.5)[1.8.9]试写出下面语言的正则定义。\\
1) 包含5个元音的所有小写字母串,这些串中元音按照顺序出现。\\
$answer \rightarrow  AEIOU$\\
$A \rightarrow (consonant|a)^*a(consonant|a)^* $\\
$E \rightarrow (consonant|e)^*e(consonant|e)^* $\\
$I\rightarrow (consonant|i)^*i(consonant|i)^* $\\
$O \rightarrow (consonant|o)^*o(consonant|o)^* $\\
$U \rightarrow (consonant|u)^*u(consonant|u)^* $\\
$consonant \rightarrow  [bcdfghjklmnpqrstvwxyz] $\\
\\
2) 所有由a和b组成的且不含子串abb的串。\\
$answer \rightarrow b^*(a+b?)^* $\\
\\
3) 所有由a和b组成的且不含子序列abb的串。\\
$answer \rightarrow b^*a^*b?a^*$\\
\end{document}