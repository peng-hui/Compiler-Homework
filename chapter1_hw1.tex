\documentclass[a4paper, 16pt]{article}
\usepackage[UTF8]{ctex}
\usepackage{geometry}
\usepackage{setspace}
\geometry{left = 1.0cm, right = 1.0cm, top = 2.0 cm, bottom = 2.5cm}
\title{编译原理作业题一}
\author{李鹏辉}

\begin{document}
\maketitle

\begin{Large}
1. 编译器想对于解释器的优点是什么?解释器相对编译器的优点是什么?\\

	编译器通过阅读源语言的程序,并把该程序翻译成为一个等价的目标语言编写的程序,而解释器不通过翻译的方式生成目标程序,而是直接利用用户提供的输入执行源程序中指定的操作。\\
	
	在用户输入映射成为输出的过程中,由编译器产生的机器语言目标程序通常要比一个解释器快很多,因为编译器可以在代码之间进行更多的优化,但是解释器的错误诊断效果通常要比编译器更好,因为它逐个语句的执行源程序,并且更加容易实现跨平台的代码。此外,解释器每次运行时,都要重新进行解释,而编译器可以单次编译,运行多次。\\

2. 在一个语言处理系统中,编译器产生的汇编语言而不是机器语言的好处是什么?\\

	机器语言通常都是二进制代码,01字符串,这样在对编译器编写和调试过程中遇到的困难会比较大。二进制代码的可读性远远小于汇编语言,给编译器的开发带来不少麻烦,而与底层硬件接轨的汇编语言,其实现了汇编指令到机器指令的映射,所以汇编器开发者直接产生汇编语言就可以实现同样的开发效果,且更加简单方便,所以产生汇编语言要优于直接产生机器语言。\\

3. 对下图的块结构,指出赋值给w, x, y, z的数值\\

	通过程序块分析和程序直接运行,得到的结果是:\\
	
	w = 13; x = 11; y = 13; z = 11;\\
	
	
4. 对下图b,指出赋值给w, x, y, z的数值\\
	
	同理可得:\\
	
	w = 9; x = 7; y = 13; z = 11;\\
	
5. 下列C代码的打印结果是什么?\\

	main函数先调用函数b,然后调用函数c,这里的宏定义在使用时会直接用'(x+1)'替换a,所以在b函数中,对全局变量x = a;其实是将(x+1)赋值给x,得到x = 3;然后在c函数中,局部变量x = 1; 打印的是a = (x+1);所以打印的结果是2;所以最终打印出来的结果是: 3 2
	
\end{Large}
\end{document} 