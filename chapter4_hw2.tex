\documentclass[a4paper, 10pt]{article}
\usepackage[UTF8]{ctex}
\usepackage{geometry}
\usepackage{graphicx}
\usepackage{setspace}
\usepackage{float}
\usepackage{listings}
\usepackage{xcolor}
\lstset{
    numbers=left, 
    numberstyle= \tiny, 
    keywordstyle= \color{ blue!70},
    commentstyle= \color{red!50!green!50!blue!50}, 
    frame=shadowbox, % 阴影效果
    rulesepcolor= \color{ red!20!green!20!blue!20} ,
    escapeinside=``, % 英文分号中可写入中文
    xleftmargin=2em,xrightmargin=2em, aboveskip=1em,
    framexleftmargin=2em
} 
\geometry{left = 1.0 cm, right = 1.0cm, top = 2.0cm, bottom = 2.0cm	}
\title{编译原理第四章(二)}
\author{李鹏辉}

\begin{document}
\maketitle 
1. 使得文法的预测分析产生回溯的原因是什么? 仅使用FIRST集合可以避免回溯吗?为什么?

因为根据输入串和非终结符的某个推导的选择过程中,字符是按照顺序读入,然后进行匹配和选择,所得到的关于字符串的信息是局部的。而我们产生某个推导选择的过程中,只能依靠当前所扫描到的字符串,因此当当前匹配不成功时,需要回溯到上一层,重新选择,进行匹配。

不能。FIRST集合仅仅是当文法满足LL(1)时,才可以根据FIRST集合做出没有回溯的唯一选择。还需要考虑FOLLOW集合等\\

2. 考虑文法:

为防止歧义,下表中使用下划线$\_$代替原有的$-$
\begin{table}[H]
\centering
\begin{tabular}{c}
\hline
$lexp \rightarrow atom | list$\\
$atom \rightarrow number | identifiler$\\
$list \rightarrow (lexp\_seq)$\\
$lexp\_seq \rightarrow leap\_seq\;lexp | lexp$\\
\hline
\end{tabular}
\end{table}
a. 消除左递归。
\begin{table}[H]
\centering
\begin{tabular}{c}
\hline
$lexp \rightarrow atom | list$\\
$atom \rightarrow number | identifiler$\\
$list \rightarrow (lexp\_seq)$\\
$lexp\_seq \rightarrow lexp \; lexp\_seq'$\\
$lexp\_seq' \rightarrow \varepsilon | lexp \; lexp\_seq'$\\
\hline
\end{tabular}
\end{table}

b. 求得该文法的FIRST集合和FOLLOW集合

\begin{table}[H]
\centering
\caption{FIRST集合}
\begin{tabular}{c}
\hline
$FIRST(number) = \{number \}$\\
$FIRST(identifiler = \{identifiler \; \}$\\
$FIRST(\;(\;) = \{ \; ( \; \}$\\
$FIRST(\; ) \;) = \{ \; )\; \}$\\
$FIRST(lexp) = \{number, identifiler, ( \; \}$\\
$FIRST(atom) = \{number, identifier \}$\\
$FIRST(list) = \{\; ( \; \}$\\
$FIRST(lexp\_seq) = \{ number, identifiler, ( \}$\\
$FIRST(lexp\_seq') = \{ \varepsilon ,number, identifiler, ( \; \}$\\
\hline
\end{tabular}
\end{table}

首先需要判断该文法的开始符号为$lexp$

\begin{table}[H]
\centering
\caption{FOLLOW集合}
\begin{tabular}{c}
\hline
$FOLLOW(lexp) =  \{ \$ ,number, identifiler, (, \;, ) \; \}$\\
$FOLLOW(lexp\_seq) = \{ \;) \} $\\
$FOLLOW(atom) = \{ \$ ,number, identifiler, (, \;, ) \;  \}$\\
$FOLLOW(list) = \{ \$ ,number, identifiler, (, \;, ) \;  \}$\\
$FOLLOW(leap\_seq') = \{  \;)  \; \}$\\
\hline
\end{tabular}
\end{table}

c. 说明所得的文法是LL(1)文法

$1.\;\; lexp \rightarrow atom | list$

$FIRST(atom) \cap FIRST(list) = \emptyset  \Rightarrow $ 满足条件\\

$2.\;\; atom \rightarrow number | identifiler$

$FIRST(number) \cap FIRST(identifiler) \emptyset \Rightarrow$ 满足条件 \\

$3.\;\;lexp\_seq' \rightarrow \varepsilon | lexp \; lexp\_seq'$ \\

$FIRST(lexp\; lexp\_seq') = FIRST(lexp)$

$FIRST(\varepsilon) = FOLLOW(lexp\_seq') = \{\;)\;\}$

$FIRST(lexp\; lexp\_seq') \cap FIRST(\varepsilon) = \emptyset \Rightarrow$ 满足条件\\

d. 构造LL(1)文法分析表
\begin{table}[H]
\centering
\caption{文法分析表}
\begin{tabular}{c|c|c|c|c|c}
\hline
非终结符 & $number$&$identifiler$ & $ ($ & $)$  & $ \$ $\\
\hline
lexp & $lexp \rightarrow atom$ & $lexp \rightarrow atom$ & $lexp \rightarrow list$ & $ $     \\
\hline 
atom & $atom \rightarrow number$ & $atom \rightarrow identifiler$ & &   \\
\hline 
list &$ $ & $ $ & $list \rightarrow (lexp\_seq)$&\\
\hline
$lexp\_seq$ &$lexp\_seq \rightarrow lexp \; lexp\_seq'$ & $lexp\_seq \rightarrow lexp \; lexp\_seq'$ &$lexp\_seq \rightarrow lexp \; lexp\_seq'$& \\
\hline
$lexp\_seq' $ &$lexp\_seq' \rightarrow lexplexp\_seq'$ &$lexp\_seq' \rightarrow lexplexp\_seq'$ &$lexp\_seq' \rightarrow lexplexp\_seq'$ & $lexp\_seq' \rightarrow \varepsilon$ \\ 
\hline
\end{tabular}
\end{table}

e. 对输入串$(a(b(2))(c))$给出相应得到LL(1)分析程序的动作
\begin{table}[H]
\centering
\caption{$(a(b(2))(c))$}
\begin{tabular}{l| r|r|l}
\hline
 已匹配 & 栈 & 输入 &动作 \\
 \hline
 &$lexp\$$ & $(a(b(2))(c))\$ $ &\\
 &$list\$$ & $(a(b(2))(c))\$ $ & 输出$lexp \rightarrow list$\\
 &$(lexp\_seq)\$$ &$(a(b(2))(c))\$ $ &输出 $list \rightarrow (lexp\_seq)$\\
 ( & $lexp\_seq)\$$ &$a(b(2))(c))\$ $ &匹配 (\\
 
 ( & $lexp \;lexp\_seq'\$$ &$a(b(2))(c))\$ $ & 输出 $lexp\_seq \rightarrow lexp \; lexp\_seq'$ \\
 ( & $atom\;lexp\_seq'\$$& $a(b(2))(c))\$ $ &输出$lexp \rightarrow atom$\\
 ( &$identifiler\;lexp\_seq'\$$ &$a(b(2))(c))\$ $ &输出 $atom \rightarrow identifilier$\\
  ( &$a\;lexp\_seq'\$$ &$a(b(2))(c))\$ $ &输出 $identifilier \rightarrow a$\\
 ( & $a\;lexp\_seq'\$$ & $a(b(2))(c))\$ $ & 匹配 a\\
 
 (a & $lexp \;lexp\_seq'\$$ & $(b(2))(c))\$ $ & 输出$lexp\_seq' \rightarrow lexp\; lexp\_seq'$\\
 (a & $list\; lexp\_seq'\$$&$(b(2))(c))\$ $& 输出$lexp \rightarrow list$ \\
 (a & $(lexp\_seq)\;lexp\_seq'\$$& $(b(2))(c))\$ $&输出$list \rightarrow (lexp\_seq)$\\
 (a( & $lexp\_seq) \; lexp\_seq'\$$ &$b(2))(c))\$ $  &匹配(\\
 
 (a( & $lexp \; lexp\_seq' ) lexp\_seq'\$$ & $b(2))(c))\$ $  &输出$lexp\_seq' \rightarrow lexp\; lexp\_seq'$ \\
 (a( & $atom \; lexp\_seq' ) lexp\_seq'\$$ & $b(2))(c))\$ $  &输出$lexp\_seq' \rightarrow lexp\; lexp\_seq'$ \\
 (a( & $identifilier \; lexp\_seq' ) lexp\_seq'\$$ & $b(2))(c))\$ $  &输出 $atom \rightarrow identifilier$ \\
  (a( & $b \; lexp\_seq' ) lexp\_seq'\$$ & $b(2))(c))\$ $  &输出 $identifilier \rightarrow b$ \\
 (a(b &$lexp\_seq' ) lexp\_seq'\$$&$(2))(c))\$ $ &匹配b\\
 
 
(a(b&$lexp \;lexp\_seq') lexp\_seq'\$$ & $(2))(c))\$ $ & 输出$lexp\_seq' \rightarrow lexp\; lexp\_seq'$\\
(a(b&$list \;lexp\_seq') lexp\_seq'\$$ & $(2))(c))\$ $ & 输出$lexp \rightarrow list$\\
(a(b&$(lexp\_seq)lexp\_seq') lexp\_seq'\$$ &$(2))(c))\$ $ &输出 $list \rightarrow (lexp\_seq)$\\
(a(b(& $lexp\_seq)lexp\_seq') lexp\_seq'\$$ &$2))(c))\$ $ &匹配 (\\
 
(a(b(& $lexp \; lexp\_seq')lexp\_seq')lexp\_seq'\$$& $2))(c))\$$&输出$lexp\_seq \rightarrow lexp\; lexp\_seq'$ \\
(a(b( & $atom \; lexp\_seq' )lexp\_seq') lexp\_seq'\$$ & $2))(c))\$ $  &输出$lexp \rightarrow atom$ \\
(a(b( & $number \;  lexp\_seq' )lexp\_seq') lexp\_seq'\$$ & $2))(c))\$ $  & 输出$atom \rightarrow number$ \\
(a(b( & $2 \;  lexp\_seq' )lexp\_seq') lexp\_seq'\$$ & $2))(c))\$ $  &输出 $number \rightarrow 2$ \\
(a(b(2 & $lexp\_seq' )lexp\_seq') lexp\_seq'\$$ & $))(c))\$ $  & 匹配2 \\


(a(b(2 &$ )lexp\_seq') lexp\_seq'\$$&$))(c))\$ $ & 弹出$lexp\_seq'$\\
(a(b(2) &$lexp\_seq') lexp\_seq'\$$&$)(c))\$ $ & 匹配)\\
(a(b(2) &$)lexp\_seq'\$$&$)(c))\$ $ & 弹出$lexp\_seq'$\\  
(a(b(2) &$  lexp\_seq'\$$&$(c))\$ $ & 匹配)\\

  
(a(b(2))&$lexp \;lexp\_seq'\$$ & $(c))\$ $ & 输出$lexp\_seq' \rightarrow lexp\; lexp\_seq'$\\
(a(b(2))&$list \;lexp\_seq'\$$ & $(c))\$ $ & 输出$lexp \rightarrow list$\\
(a(b(2))&$(lexp\_seq)lexp\_seq'\$$ &$(c))\$ $ &输出 $list \rightarrow (lexp\_seq)$\\
(a(b(2))(& $lexp\_seq)lexp\_seq'\$$ &$c))\$ $ &匹配 (\\
 
(a(b(2))& $lexp \; lexp\_seq' ) lexp\_seq'\$$ & $c))\$ $  &输出$lexp\_seq \rightarrow lexp\; lexp\_seq'$ \\
(a(b(2))(& $atom \; lexp\_seq' ) lexp\_seq'\$$ & $c))\$ $  &输出$lexp \rightarrow atom$ \\
(a(b(2))(& $identifilier \; lexp\_seq' ) lexp\_seq'\$$ & $c))\$ $  & 输出$atom \rightarrow identifilier$ \\
(a(b(2))(& $c \; lexp\_seq' ) lexp\_seq'\$$ & $c))\$ $  & 输出$identifilier \rightarrow c$ \\
(a(b(2))(c &$lexp\_seq' ) lexp\_seq'\$$&$))\$ $ &匹配c\\

(a(b(2))(c &$) lexp\_seq'\$$&$)\$ $ & 弹出$lexp\_seq'$\\  
(a(b(2))(c) &$ lexp\_seq'\$$&$\$ $ & 匹配)\\
(a(b(2))(c) &$lexp\_seq'\$$&$\$$ & 弹出$lexp\_seq'$\\  
\hline
\end{tabular}
\end{table}
\end{document}
